% !TEX program = xelatex

\documentclass[10pt,a4paper]{article}
\usepackage[top = 1.5cm, bottom = 1.5cm, left = 1.5cm, right = 1.5cm]{geometry}

\usepackage{titling}
\usepackage[czech]{babel}
\usepackage{graphicx}
\usepackage{lmodern}
\usepackage{hyperref}
\usepackage{setspace}
\usepackage{csvsimple}

\usepackage{amsmath}
\usepackage{amssymb}
\usepackage{gensymb}
\usepackage{units}
\usepackage{bm}
\usepackage{bbm}
\delimitershortfall=-1pt

\usepackage{mathtools}
\usepackage{accents}
\usepackage{calc}

% no page break
\newenvironment{absolutelynopagebreak}
  {\par\nobreak\vfil\penalty0\vfilneg
   \vtop\bgroup}
  {\par\xdef\tpd{\the\prevdepth}\egroup
   \prevdepth=\tpd}


% redefine \sqrt
\usepackage{letltxmacro}
\makeatletter
\let\oldr@@t\r@@t
\def\r@@t#1#2{%
\setbox0=\hbox{$\oldr@@t#1{#2\,}$}\dimen0=\ht0
\advance\dimen0-0.2\ht0
\setbox2=\hbox{\vrule height\ht0 depth -\dimen0}%
{\box0\lower0.4pt\box2}}
\LetLtxMacro{\oldsqrt}{\sqrt}
\renewcommand*{\sqrt}[2][\ ]{\oldsqrt[#1]{#2\,}\,}
\makeatother

% redefine \hbar
\LetLtxMacro{\oldhbar}{\hbar}
\renewcommand*{\hbar}{{\mathpalette\hbaraux\relax\mathrm{h}}}
\newcommand*{\hbaraux}[2]{\sbox0{\mathsurround=0pt$#1\mathchar'26$}\mkern-1mu\lower.07\ht0\box0\mkern-8mu}

% define \bigcdot
\makeatletter
\newcommand*\bigcdot{\mathpalette\bigcdot@{.8}}
\newcommand*\bigcdot@[2]{\mathbin{\vcenter{\hbox{\scalebox{#2}{$\m@th#1\bullet$}}}}}
\makeatother

\def\ph{\phantom}
\def\vph{\vphantom}
\def\hph{\hphantom}
\def\rzw{\mathrlap}
\def\lzw{\mathllap}
\def\czw{\mathclap}

\newcommand{\nph}[1]{\settowidth{\dimen0}{#1}\hspace*{-\dimen0}}

\newcommand*{\mask}[2]{%
    \mathord{\makebox[\widthof{\(#1\)}]{\(#2\)}}%
}

\def\?{\mathit{?}}

\newcommand{\comm}[2]{\left[ #1, #2 \right]}
\newcommand{\const}[1]{\text{#1}}
\newcommand{\norm}[1]{\left\lVert#1\right\rVert}

\newcommand{\mat}[1]{
    \begin{pmatrix}
        #1
    \end{pmatrix}
}

\newcommand{\mata}[2]{
    \left(
    \begin{array}{@{}#1@{}}
        #2
    \end{array}
    \right)
}

\newcommand{\smat}[2][1]{
    \scalebox{#1}{$\mat{#2}$}
}

\renewcommand{\d}[1]{\;\const{d}#1}
\newcommand{\dd}[2]{\frac{\const{d} #1}{\const{d} #2} \;}
\newcommand{\pd}[2]{\frac{\partial  #1}{\partial  #2} \;}

\newcommand{\bra}[1]{\left< #1 \right|}
\newcommand{\ket}[1]{\left| #1 \right>}
\newcommand{\braket}[2]{\left< #1 \middle| #2 \right>}

\newcommand{\e}[1]{\const{e}^{#1}}
\renewcommand{\i}{\const{i}}

\newcommand{\bigdot}[1]{\accentset{\bigcdot}{#1}}
\newcommand{\bigddot}[1]{\accentset{\bigcdot\bigcdot}{#1}}
\newcommand{\vechat}[1]{\hat{\vec{#1}}}

\def\tg{\operatorname{tg}}
\def\ctg{\operatorname{ctg}}
\def\Span{\operatorname{span}}

\def\kzero{\ket{\mask{+}{0}}}
\def\kone{\ket{\mask{+}{1}}}
\def\ktwo{\ket{\mask{+}{2}}}
\def\kplus{\ket{+}}
\def\kminus{\ket{-}}

\def\bone{\bra{\mask{+}{1}}}
\def\btwo{\bra{\mask{+}{2}}}

\def\R{\mathbb{R}}
\def\C{\mathbb{C}}
\def\1{\mathbbm{1}}
\def\0{\mask{+}{0}}

\def\Parity{\hat{\mathcal P}}
\def\Cop{\hat{\mathcal C}}
\def\G{\mathnormal\Gamma}


\begin{document}

\title{Úvod do kvantové mechaniky: Domácí úkoly z přednášek}
\author{Michal Grňo}
\date{\today}

\maketitle

\section{Projekce spinu do obecného směru}

\subsection{Zadání}
Nechť projekce spinu do osy $z$ je $\nicefrac{1}{2}$. S jakou pravděpodobností naměříme projekci spinu $\pm\nicefrac{1}{2}$ do obecného směru?

\subsection{Řešení}
Zavedeme si jednotkový vektor $\vec n$ parametrizovaný sférickými souřadnicemi:
\begin{equation*}
    \vec n = \mat{
        \sin \vartheta \cos \varphi \\
        \sin \vartheta \sin \varphi \\
        \cos \vartheta
    }
\end{equation*}
Zavedeme si operátor $\hat S_{\vec n} = \vec n \cdot \vechat S$, kde $\vechat S$ reprezentujeme Pauliho maticemi:
\begin{equation*}
    \vechat S = \frac{1}{2} \mat{
        \smat[0.8]{0 & \ph{-} 1 \\ 1 & \ph{-} 0} \\[10pt]
        \smat[0.8]{0 & -\i \\ \i & \ph{-} 0} \\[10pt]
        \smat[0.8]{1 & \ph{-} 0 \\ 0 & -1}
    }
\end{equation*}
Operátor $\hat S_{\vec n}$ nám potom vyjde:
\begin{equation*}
    \hat S_{\vec n} = \frac{1}{2} \mat{
        \cos \vartheta & \sin \vartheta \; \e{-\i \varphi} \\
        \sin \vartheta \; \e{\i \varphi} & -\cos \vartheta
    }
\end{equation*}
Víme, že vlastní čísla $\hat S_{\vec n}$ jsou $\pm \nicefrac{1}{2}$, přejdeme tedy rovnou k nalezení vlastních vektorů:
\begin{equation*}
    \ker(\hat S_{\vec n} - \nicefrac{1}{2} \, \hat I)
    = \ker\mat{
        \cos \vartheta - 1 & \sin \vartheta \; \e{-\i \varphi} \\
        \sin \vartheta \; \e{\i \varphi} & -\cos \vartheta - 1
    }
    = \operatorname{span} \{ \mat{
        \e{-\i \varphi} (\cot \vartheta + \csc \vartheta) \\ 1
    } \}
\end{equation*}
\begin{equation*}
    \ker(\hat S_{\vec n} + \nicefrac{1}{2} \, \hat I)
    = \ker\mat{
        \cos \vartheta + 1 & \sin \vartheta \; \e{-\i \varphi} \\
        \sin \vartheta \; \e{\i \varphi} & -\cos \vartheta + 1
    }
    = \operatorname{span} \{ \mat{
        \e{-\i \varphi} (\cot \vartheta - \csc \vartheta) \\ 1
    } \}
\end{equation*}
Normalizované vlastní stavy jsou tedy:
\begin{equation*}
    \ket{\pm \vec n} =
    \frac{1}{\oldsqrt{2}} \,
    \frac{1}{\sqrt{1 \pm \cos \vartheta}}
    \mat{
        \cos \vartheta \pm 1 \\
        \e{\i \varphi} \sin \vartheta
    }
\end{equation*}
Pravděpodobnost naměření $\ket{\pm \vec n}$, je-li stav $\ket{+z}$, je:
\begin{equation*}
    P = | \braket{+z}{\pm \vec n} |^2
    =
    \left|
    \mat{1 \\ 0}
    \cdot
    \frac{1}{\oldsqrt{2}} \,
    \frac{1}{\sqrt{1 \pm \cos \vartheta}}
    \mat{
        \cos \vartheta \pm 1 \\
        \e{\i \varphi} \sin \vartheta
    }
    \right|^2
    =
    \left|
    \frac{1}{\oldsqrt{2}} \,
    \frac{\cos \vartheta \pm 1}{\sqrt{1 \pm \cos \vartheta}}
    \right|^2
    = \frac{1}{2} \pm \frac{1}{2} \cos \theta.
\end{equation*}


\section{Rabiho metoda}

\subsection{Zadání}
Mějme částici se spinem $\nicefrac{1}{2}$ v poli s intenzitou
\begin{equation*}
    \vec B = \mat{
        B_1 \cos \omega t \\
        B_1 \sin \omega t \\
        B_0
    },
\end{equation*}
kde $B_1 \ll B_0$, $\omega \approx -K B_0$.

Stav spinu $\ket{\psi(t)}$ začíná v čase $t=0$ jako $\ket{\pm z}$. S jakou pravděpodobností bude v obecném čase $t$ ve stavu $\ket{-z}$?

\subsection{Řešení}
Hamiltonián systému je
\begin{equation*}
    \hat H = - K \; \vechat S \cdot \vec B,
\end{equation*}
kde $\vechat S$ reprezentujeme Pauliho maticemi. Využijeme rozklad $\vechat S$ na žebříkové operátory $\hat S_\pm$:
\begin{equation*}
    \hat S_\pm
    = \hat S_\mathrm{x} \pm \i \hat S_\mathrm{y}
    = \frac{1}{2} \mat{ 0 & 1 \pm 1 \\ 1 \mp 1 & 0 }
    = \begin{Bmatrix}
        \smat[0.8]{0 & 1 \\ 0 & 0} \\[15pt]
        \smat[0.8]{0 & 0 \\ 1 & 0}
    \end{Bmatrix}
    = \ket{\pm} \bra{\mp}.
\end{equation*}
Navíc víme, že
\begin{equation*}
    \hat S_\mathrm{z}
    = \frac{1}{2} (\ket{+} \! \bra{+} \, - \, \ket{-} \! \bra{-}).
\end{equation*}
Podobně rozložíme $\vec B$:
\begin{equation*}
    B_\pm
    = B_\mathrm{x} \pm \i B_\mathrm{y}
    = B_1 (\cos \omega t \pm \i \sin \omega t)
    = B_1 \, \e{\pm \i \omega t}.
\end{equation*}
Nyní můžeme vyjádřit hamiltonián ve tvaru
\begin{equation*}
    \hat H
    = -K \; \vechat S \cdot \vec B
    = -K \; \left(
        \frac{1}{2} (\hat S_+ B_- + \hat S_- B_+)
        + \hat S_\mathrm{z} B_\mathrm{z}
    \right)
    = -\frac{K}{2} \left(
        B_1 \, \e{-\i \omega t} \ket{+} \! \bra{-} \, + \,
        B_1 \, \e{+\i \omega t} \ket{-} \! \bra{+} \, + \,
        \ket{+} \! \bra{+} \, - \, \ket{-} \! \bra{-}
    \right),
\end{equation*}
tedy v maticové formě
\begin{equation*}
    \bra{\pm} \hat H \ket{\pm} =
    -\frac{K}{2} \mat{
        B_0 & B_1 \e{-\i \omega t}  \\
        B_1 \e{+\i \omega t} & -B_0
    }.
\end{equation*}

\bigskip

Nyní se můžeme pustit do řešení samotné Schrödingerovy rovnice.
\begin{equation*}
    -\i \dd{}{t}\!\! \ket{\psi} = \hat H(t) \ket{\psi}
\end{equation*}
\begin{equation*}
    -\i \dd{}{t} \mat{ c_+(t) \\ c_-(t) }
    = -\frac{K}{2} \mat{
        B_0 & B_1 \e{-\i \omega t}  \\
        B_1 \e{+\i \omega t} & -B_0
    }
    \mat { c_+(t) \\ c_-(t) }
\end{equation*}
\begin{align}
    -\i\bigdot c_+
    &= - \frac{K B_0}{2} \; \, c_+
    - \frac{K B_1}{2} \; \, \e{-\i \omega t} \, c_-
    \label{rce_cplus}
    \\
    -\i\bigdot c_-
    &= + \frac{K B_0}{2} \; \, c_-
    - \frac{K B_1}{2} \; \, \e{+\i \omega t} \, c_+
    \label{rce_cminus}
\end{align}
\begin{gather*}
    \text{Z rovnice \eqref{rce_cplus}: }
    \quad
    c_- = \frac{2}{K B_1} \e{+\i \omega t} \left( \i \bigdot c_+ - \frac{K B_0}{2} \, c_+ \right)
    = \e{+\i \omega t} \left(
        \i \; \frac{2}{K B_1} \, \bigdot{c}_+
        - \frac{B_0}{B_1} \, c_+
    \right)
    \\
\end{gather*}
\begin{gather*}
    \text{Z rovnice \eqref{rce_cminus}: }
    \\
    -\i \dd{}{t}
    \e{+\i \omega t}
    \left(
        \i \; \frac{2}{K B_1} \, \bigdot{c}_+
        - \frac{B_0}{B_1} \, c_+
    \right)
    =
    \frac{K B_0}{2} \; \, \e{+\i \omega t}
    \left(
        \i \; \frac{2}{K B_1} \, \bigdot{c}_+
        - \frac{B_0}{B_1} \, c_+
    \right)
    - \frac{K B_1}{2} \; \, \e{+\i \omega t} \, c_+
    \\[10pt]
    \big\Downarrow
    \\[10pt]
    0 = \bigddot c_+ + \i\omega \bigdot c_+ +
    \underbrace{\left(
        \frac{ {B_0}^2 K^2 }{4}
        - \frac{ B_0 K \omega }{2}
        - \frac{ {B_1}^2 K^2 }{4}
    \right)}_\kappa c_+
\end{gather*}
Máme tedy rovnici typu
\begin{align*}
    f'' + \i\omega f' + \kappa f &= 0 \\
    \lambda^2 + \i\omega\rzw{\lambda}\ph{f'} + \kappa\ph{f} &= 0
\end{align*}
\begin{gather*}
    \lambda
    = \frac{-\i \omega \pm \sqrt{(\i\omega)^2 - 4 \kappa}}{2}
    = - \frac{\i}{2} \, \omega \pm \frac{\i}{2} \sqrt{\omega^2 + 4\kappa}
    \\[10pt]
    f
    = C_1 \, \exp\i(-\frac{\omega}{2} + \frac{1}{2} \sqrt{\omega^2 + 4\kappa}) t
    + C_2 \, \exp\i(-\frac{\omega}{2} - \frac{1}{2} \sqrt{\omega^2 + 4\kappa}) t
\end{gather*}
Odmocninu můžeme ještě dále zjednodušit zanedbáním členu s ${B_1}^2$, který je výrazně menší než ostatní členy (viz zadání).
\begin{gather*}
    \sqrt{\omega^2 + 4\kappa}
    = \sqrt{
        \omega^2
        + {B_0}^2 K^2
        - 2 B_0 K \omega
        - {B_1}^2 K^2
    } \approx  \sqrt{
        \omega^2
        - 2 B_0 K \omega
        + {B_0}^2 K^2
    }
    = B_0 K - \omega
\end{gather*}
Pro $c_+$ tedy dostáváme:
\begin{align*}
    c_+(t) &= \e{-\i\omega t/2} \left(
        C_1 \e{+\i t\, \frac{B_0 K - \omega}{2}} +
        C_2 \e{-\i t\, \frac{B_0 K - \omega}{2}}
    \right)
    \\[10pt]
    c_+(t) &= \e{-\i\omega t/2} \left(
        D_1 \cos \frac{B_0 K - \omega}{2} t +
        D_2 \sin \frac{B_0 K - \omega}{2} t
    \right)
\end{align*}
Dosazením do \eqref{rce_cminus} získáme:
\begin{gather*}
    c_-(t) = \e{+\i \omega t} \e{-\omega t/2} \left(
        \i \; \frac{2}{K B_1} \, \dd{}{t} \left(
            D_1 \cos \tfrac{B_0 K - \omega}{2} t +
            D_2 \sin \tfrac{B_0 K - \omega}{2} t
        \right)
        - \frac{B_0}{B_1} \, \left(
            D_1 \cos \tfrac{B_0 K - \omega}{2} t +
            D_2 \sin \tfrac{B_0 K - \omega}{2} t
        \right)
    \right)
\end{gather*}
Konstanty $D_n$ určíme z počáteční podmínky $\ket{\psi(t=0)} = \ket{\pm z}$ a z požadavku, aby byl stav $\ket{\psi(t)}$ normalizovaný. Pro přehlednost si zavedeme označení $\psi_\pm(0) \equiv \ket{\pm z}$.
\begin{gather*}
    1 = \braket{\pm z}{\psi_\pm(0)}
    = \begin{Bmatrix}
        \smat[0.8]{1 \\ 0} \\[15pt]
        \smat[0.8]{0 \\ 1}
    \end{Bmatrix}
    \cdot \mat{
        D_1 \\ D_3
    }
\end{gather*}
Tedy pro $\psi_+$ máme $D_1 = 1, \; D_2 = 0$, pro $\psi_-$ zase $D_3 = 1, \; D_4 = 0$. Zbylé konstanty dopočítáme dosazením. Celkově platí:
\begin{equation*}
    \ket{\psi_+(t)} = \mat{
        \ph{\i \,} \e{\displaystyle -\i\omega t/2} \;\; \cos \dfrac{B_0 K - \omega}{2} t
        \\[10pt]
        \i \, \e{\displaystyle +\i\omega t/2} \;\; \sin \dfrac{B_0 K - \omega}{2} t
    } \: ,
    \quad
    \ket{\psi_-(t)} = \mat{
        \i \, \e{\displaystyle -\i\omega t/2} \;\; \sin \dfrac{B_0 K - \omega}{2} t
        \\[10pt]
        \ph{\i \,} \e{\displaystyle +\i\omega t/2} \;\; \cos \dfrac{B_0 K - \omega}{2} t
    } \: .
\end{equation*}
Pokud by nastal případ $\omega = -B_0 K$, máme:
\begin{equation*}
    \ket{\psi_+(t)} = \mat{
        \ph{\i \,} \e{\displaystyle +\i K B_0 t/2} \;\; \cos B_0 K t
        \\[10pt]
        \i \, \e{\displaystyle -\i K B_0 t/2} \;\; \sin B_0 K t
    } \: ,
    \quad
    \ket{\psi_-(t)} = \mat{
        \i \, \e{\displaystyle +\i K B_0 t/2} \;\; \sin B_0 K t
        \\[10pt]
        \ph{\i \,} \e{\displaystyle -\i K B_0 t/2} \;\; \cos B_0 K t
    } \: .
\end{equation*}



\section{\texorpdfstring{$\mathcal{PT}$}{PT}-symetrický hamiltonián}
\subsection{Zadání}
Máme zadány operátory
\begin{equation*}
    \hat\G = \smat[0.8]{
        \ph{-}K & -\i a \\
        - \i a  & -K
    },
    \quad
    \Parity = \smat[0.8]{
        1 & \ph{-}0 \\
        0 & -1
    },
\end{equation*}
kde $K, a \in \R_+$. Ukažte, že
\begin{enumerate}
    \item $\Parity^2 = \1, \quad \hat\G^+ = \Parity \; \hat\G \; \Parity$
    \item $\hat\G \ket{\pm} = \pm \G \ket{\pm}, \quad \G = \sqrt{K^2 - a^2}, \quad$ tedy pro $a<K$ je neporušena $\mathcal{PT}$-symetrie.
    \item pro $a>K$ platí $\bra{\pm} \Parity \ket{\pm} = 0$
    \item pro $a<K$ platí $\bra{m} \Parity \ket{n} = (-1)^m \, \delta_{mn}$
\end{enumerate}

Pro $\mathcal{PT}$-symetrické operátory platí upravená relace úplnosti
\begin{equation*}
    \1 = \sum_n (-1)^n \ket{n} \bra{n} \Parity .
\end{equation*}
Ověřte její platnost pro $\hat\G$ při $a<K$.

Definujeme operátor
\begin{equation*}
    \Cop \coloneqq \sum_n \ket{n} \bra{n} \Parity,
\end{equation*}
ten komutuje s $\hat\G$ i $\Parity$ a tvoří základ skalárního součinu, pod kterým je $\ket{n}$ ortonormální systém:
\begin{equation*}
    (\psi, \phi)_\mathcal{CPT} \coloneqq \bra{\psi} \Parity \Cop \ket{\phi} .
\end{equation*}
Vypočtěte $\Cop$ a pro $a<K$ ověřte, že
\begin{enumerate}
    \item $\Cop^2 = \1$
    \item $\Cop \ket{\pm} = \pm \ket{\pm}$
    \item $\Cop \ket{+}\bra{+} \Parity + \Cop \ket{-}\bra{-} \Parity = \1$
    \item $\bra{m} \Parity \Cop \ket{n} = \delta_{mn}$
\end{enumerate}

\subsection{Řešení}
\begin{gather*}
    \Parity^2 = \smat[0.8]{1 \\& -1}^2 = \smat[0.8]{1\\&1}
    \\\\
    \Parity \; \hat\G \; \Parity =
    \smat[0.8]{1 \\& -1}
    \smat[0.8]{\ph{-}K & -\i a \\ -\i a & -K}
    \smat[0.8]{1 \\& -1}
    =
    \smat[0.8]{K & \ph{-} \i a \\ \i a & -K}
    = \hat \G^+
    \\\\
    0 = \left|\hat\G - \lambda\1\right| = \begin{vmatrix}
        K-\lambda & -\i a \\ -\i a & -K-\lambda
    \end{vmatrix}
    = -(K-\lambda)(K+\lambda) + a^2 = \lambda^2 - (K^2 - a^2)
    \\
    \lambda = \pm \sqrt{K^2 - a^2} \equiv \pm \G
\end{gather*}
Nyní nalezneme vlastní vektory pro $a>K$, využijeme parametrizaci $a = K \cosh t, \; t \in \R$.
\begin{gather*}
    \hat\G = K \mat{
        1 & -\i \cosh t \\
        -\i \cosh t & -1
    } \: ,
    \quad
    \G = \sqrt{K^2 - K^2 \cosh^2 t}
    = \i K \sinh t \: ,
    \quad
    \ket{\pm} = \mat{ A_\pm \\ B_\pm }
    \\
    \hat\G\ket{\pm} = \pm\G\ket{\pm}
    \\[10pt]
    \ph{-}A_\pm - \i B_\pm \cosh t = \pm \i A_\pm \sinh t \\
    -B_\pm - \i A_\pm \cosh t = \pm \i B_\pm \sinh t
    \\[10pt]
    B_\pm = A_\pm \frac{-\i \pm \sinh t}{\cosh t}
\end{gather*}
\begin{align*}
    \bra{\pm} \Parity \ket{\pm}
    &= \smat[0.8]{A_\pm \\ B_\pm}^{\!+} \;
    \smat[0.8]{ 1 \\ & -1} \;\;
    \smat[0.8]{A_\pm \\ B_\pm}
    = {A_\pm}^2
    \smat[0.8]{1 \\ \tfrac{-\i \pm \sinh t}{\cosh t}}^{\!+} \;
    \smat[0.8]{ 1 \\ & -1} \;\;
    \smat[0.8]{1 \\ \tfrac{-\i \pm \sinh t}{\cosh t}}
    = {A_\pm}^2 \left(
        1 +
        \frac{\, \i \pm \sinh t \,}{\cosh t} \;
        \frac{\, \i \mp \sinh t \,}{\cosh t}
    \right)
    \\
    &= {A_\pm}^2 \left(
        1 +
        \frac{\i^2 - \sinh^2 t}{\cosh^2 t}
    \right)
    = {A_\pm}^2 \left(
        1 -
        \frac{1 + \sinh^2 t}{\cosh^2 t}
    \right)
    = 0
\end{align*}
Tím jsme dokázali první tři body zadání. Nyní budeme pracovat s $a<K$, zvolíme parametrizaci $a = K \sin t, \; t \in (0, \pi)$.
\begin{gather*}
    \hat\G = K \mat{
        1 & -\i \sin t \\
        -\i \sin t & -1
    } \: ,
    \quad
    \G = K\sqrt{1 - \sin^2 t} = K \cos t \: ,
    \quad
    \ket{\pm} = \mat{ A_\pm \\ B_\pm }
    \\
    \hat\G\ket\pm = \pm\G\ket\pm
    \\[10pt]
    \ph{-}A_\pm - \i B_\pm \sin t = \pm A_\pm \cos t \\
    -B_\pm - \i A_\pm \sin t = \pm B_\pm \cos t
    \\[10pt]
    B_\pm = \i A_\pm \; \frac{-1 \pm \cos t}{\sin t}
    \\[10pt]
    \ket\pm = A_\pm \mat{ 1 \\ \i \; \frac{-1 \pm \cos t}{\sin t}} = \frac{1}{\sqrt{2\cos t}} \mat{
        \frac{\i \sin t}{\sqrt{1 \mp \cos t}} \\[7pt]
        \sqrt{1 \mp \cos t}
    }
\end{gather*}
Dokážeme čtvrtý bod zadání:
\begin{align*}
    \bra{\pm} \Parity \ket{\pm} &=
    \frac{1}{2\cos t}
    \mat{
        \frac{\i \sin t}{\sqrt{1 \mp \cos t}} \\[7pt]
        \sqrt{1 \mp \cos t}
    }^{\!+}
    \mat{ 1 \\ & -1 }
    \mat{
        \frac{\i \sin t}{\sqrt{1 \mp \cos t}} \\[7pt]
        \sqrt{1 \mp \cos t}
    }
    =
    \frac{1}{2\cos t} \left(
        \frac{\sin^2 t}{1 \mp \cos t} - (1 \mp \cos t)
    \right)
    = \frac{\pm 2 \cos t}{2 \cos t}
    = \pm 1
    \\[10pt]
    \bra{\pm} \Parity \ket{\mp} &=
    \frac{1}{2\cos t}
    \mat{
        \frac{\i \sin t}{\sqrt{1 \mp \cos t}} \\[7pt]
        \sqrt{1 \mp \cos t}
    }^{\!+}
    \mat{ 1 \\ & -1 }
    \mat{
        \frac{\i \sin t}{\sqrt{1 \pm \cos t}} \\[7pt]
        \sqrt{1 \pm \cos t}
    }
    =
    \frac{1}{2\cos t} \left(
        \frac{\sin^2 t}{\sqrt{1 - \cos^2 t}} - \sqrt{1 - \cos^2 t}
    \right)
    = 0
\end{align*}
Nyní přistoupíme k odvození relace úplnosti a skalárního součinu.
\begin{gather*}
    \sum_n (-1)^n \ket{n} \bra{n} \Parity
    \; = \;
    \left(
        \ket{+} \! \bra{+}
        \; - \;
        \ket{-} \! \bra{-}
    \right)
    \Parity
    \\
    (2 \cos t)
    \ket\pm \! \bra\pm \Parity =
    \mat{
        \frac{\i \sin t}{\sqrt{1 \mp \cos t}} \\[7pt]
        \sqrt{1 \mp \cos t}
    } \!
    \mat{
        \frac{\i \sin t}{\sqrt{1 \mp \cos t}} \\[7pt]
        \sqrt{1 \mp \cos t}
    }^{\!+} \!\!
    \mat{ 1 \\ & \hspace{-.8em} -1 }
    = \mat{
        \frac{\sin^2 t}{1 \mp \cos t} & \i \sin t \\[7pt]
        -\i \sin t & 1 \mp \cos t
    } \! \mat{ 1 \\ & \hspace{-.8em} -1 }
    = \mat{
        \frac{\sin^2 t}{1 \mp \cos t} & -\i \sin t \\[7pt]
        -\i \sin t & -1 \pm \cos t
    }
\end{gather*}
\begin{align*}
    \sum_n (-1)^n \ket{n} \bra{n} \Parity
    &=
    \frac{1}{2\cos t} \smat[0.9]{
        \frac{\sin^2 t}{1 - \cos t} & -\i \sin t \\[7pt]
        -\i \sin t & -1 + \cos t
    }
    -
    \frac{1}{2\cos t} \smat[0.9]{
        \frac{\sin^2 t}{1 + \cos t} & -\i \sin t \\[7pt]
        -\i \sin t & -1 - \cos t
    }
    = \frac{1}{2\cos t} \smat[0.9]{
        \frac{\sin^2 t (2 \cos t)}{1 - \cos^2 t} & 0 \\
        0 & 2 \cos t
    }
    = \smat[0.9]{1 \\ & 1}
    \\[10pt]
    \Cop = \sum_n \ket{n} \bra{n} \Parity
    &=
    \frac{1}{2\cos t} \smat[0.9]{
        \frac{\sin^2 t}{1 - \cos t} & -\i \sin t \\[7pt]
        -\i \sin t & -1 + \cos t
    }
    +
    \frac{1}{2\cos t} \smat[0.9]{
        \frac{\sin^2 t}{1 + \cos t} & -\i \sin t \\[7pt]
        -\i \sin t & -1 - \cos t
    }
    =
    \mat{
        \sec t & -\i \tg t \\
        -\i \tg t & -\sec t
    }
\end{align*}
Přistoupíme k poslední části, kterou je ověření vlastností operátoru $\Cop$.
\begin{align*}
    \Cop^2 =
    \mat{
        \sec t & -\i \tg t \\
        -\i \tg t & -\sec t
    }^2
    =
    \mat{
        \sec^2 t - \tg^2 t & -\i \tg t \sec t + \i \tg t \sec t \\
        \i \tg t \sec t + \i \tg t \sec t & - \tg^2 t + \sec^2 t
    }
    =
    \mat{
        \frac{1 - \sin^2 t}{\cos^2 t} & 0 \\
        0 & \frac{1 - \sin^2 t}{\cos^2 t}
    }
    =
    \mat{ 1 \\ & 1}
\end{align*}
\begin{align*}
    \sqrt{2\cos t} \Cop \ket\pm
    =
    \mat{
        \sec t & -\i\tg t \\
        -\i\tg t & -\sec t
    }
    \mat{
        \frac{\i \sin t}{\sqrt{1 \mp \cos t}} \\[7pt]
        \sqrt{1 \mp \cos t}
    }
    =
    \mat{
        \i \; \frac{\sin t}{\cos t} \; \frac{1 - 1 \mp \cos t}{\sqrt{1 \mp \cos t}} \\[7pt]
        \frac{\sin^2 t - 1 \mp \cos t}{\cos t \sqrt{1 \mp \cos t}}
    }
    =
    \mat{
        \frac{\pm \i \sin t}{\sqrt{1 \mp \cos t}} \\[7pt]
        \pm \sqrt{1 \mp \cos t}
    }
    =
    \pm \sqrt{2\cos t} \ket\pm \: ,
\end{align*}
tedy $\Cop\ket\pm=\pm\ket\pm$. Poslední dva body plynou triviálně z tohoto výsledku a již dokázaných tvrzení.
\begin{gather*}
    \Cop\ket{+}\!\bra{+}\Parity + \Cop\ket{-}\!\bra{-}\Parity
    = \ket{+}\!\bra{+}\Parity - \ket{-}\!\bra{-}\Parity = \1
    \\
    \bra{m} \Parity \Cop \ket{n} = \bra{m} \Parity (-1)^n \ket{n} = (-1)^n \, (-1)^m \, \delta_{mn} = \delta_{mn}
\end{gather*}


\section{Hyperjemné štěpení (spin-spinová interakce), spin-1, pozitronium}
\subsection{Zadání}
Máme atom vodíku, složený z elektronu se spinem $\nicefrac{1}{2}$ a protonu se spinem $\nicefrac{1}{2}$. Hamiltonián systému je
\begin{equation*}
    \hat H
    = A \, \vechat S^\const{e} \!\cdot\! \vechat S^\const{p}
    - K_e \, \vechat S^\const{e} \!\cdot\! \vec B
    - K_p \, \vechat S^\const{p} \!\cdot\! \vec B
    \: , \quad
    \vec B = (0, 0, B)
    \: .
\end{equation*}
Nalezněte energie stacionárních stavů.

Dále uvažujte, spin-$1$ částici se spinem orientovaným ve směru osy $z$ a vypočtěte $\left| \braket{+z}{0 \vec{n}} \right|^2$ (tj. pravděpodobnost naměření spinu $0$ podél obecné osy $\vec n$) a $\bra{0z} \hat P_{\pm x} \ket{0z}$ (tedy pravděpodobnost naměření $0z$ po průchodu přístrojem, který změří spin ve směru $x$ a odstíní stav $0x$).

Nakonec mějme pozitronium (proton v atomu vodíku nahradíme pozitronem) s hamiltoniánem
\begin{equation*}
    \hat H
    = A \, \vechat S^{\const{e}^-} \!\! \cdot \vechat S^{\const{e}^+}
    + B \hat S^2
    \: ,
\end{equation*}
nalezněte energie stacionárních stavů.

\subsection{Řešení}
Budeme pracovat v součinové bázi vlastních stavů operátorů $\hat S_{\!z}^\const{e}$ a $\hat S_{\!z}^\const{p}$:
\begin{equation*}
    \ket{\pm\pm} \coloneqq \ket{\pm z_\const{e}} \otimes \ket{\pm z_\const{p}} \: .
\end{equation*}
Provedeme polární rozklad hamiltoniánu a vyjádříme ho v bázi $\ket{\pm\pm}$:
\begin{equation*}
    \hat H
    = A \, \vechat S^{\const{e}^+} \cdot \vechat S^{\const{e}^-} + B \hat S^2
    = A \, \frac{1}{2} \! \left( \hat S_{\!+}^\const{e} \hat S_{\!-}^\const{p} + \hat S_{\!-}^\const{e} \hat S_{\!+}^\const{p} \right) + A \,\hat S_{\!z}^\const{e} \, \hat S_{\!z}^\const{p} - K_\const{e} \, B \, \hat S_{\!z}^\const{e} - K_\const{p} \, B \, \hat S_{\!z}^\const{p}
\end{equation*}
\begin{align*}
    \hat H \ket{++}
    &= \left(A \, \frac{1}{2} \! \left( \hat S_{\!+}^\const{e} \hat S_{\!-}^\const{p} + \hat S_{\!-}^\const{e} \hat S_{\!+}^\const{p} \right) + A \,\hat S_{\!z}^\const{e} \, \hat S_{\!z}^\const{p} - K_\const{e} \, B \, \hat S_{\!z}^\const{e} - K_\const{p} \, B \, \hat S_{\!z}^\const{p}\right) \ket{++} \\
    &= \left( A \, \frac{1}{2} \, 0 + A \,\tfrac{1}{2} \, \tfrac{1}{2} - K_\const{e} \, B \, \tfrac{1}{2} - K_\const{p} \, B \, \tfrac{1}{2} \right) \ket{++}
    = \left( \frac{A}{4} - \frac{B}{2} \left( K_\const{e} + K_\const{p}\right) \right) \ket{++}
    \\\\
    \hat H \ket{--}
    &= \left(A \, \frac{1}{2} \! \left( \hat S_{\!+}^\const{e} \hat S_{\!-}^\const{p} + \hat S_{\!-}^\const{e} \hat S_{\!+}^\const{p} \right) + A \,\hat S_{\!z}^\const{e} \, \hat S_{\!z}^\const{p} - K_\const{e} \, B \, \hat S_{\!z}^\const{e} - K_\const{p} \, B \, \hat S_{\!z}^\const{p}\right) \ket{--} \\
    &= \left( A \, \frac{1}{2} \, 0 + A \left( -\tfrac{1}{2} \right) \left( -\tfrac{1}{2} \right) - K_\const{e} \, B \left( -\tfrac{1}{2} \right) - K_\const{p} \, B \left( -\tfrac{1}{2} \right) \right) \ket{--}
    = \left( \frac{A}{4} + \frac{B}{2} \left( K_\const{e} + K_\const{p}\right) \right) \ket{--}
    \\\\
    \hat H \ket{+-}
    &= \left(A \, \frac{1}{2} \! \left( \hat S_{\!+}^\const{e} \hat S_{\!-}^\const{p} + \hat S_{\!-}^\const{e} \hat S_{\!+}^\const{p} \right) + A \,\hat S_{\!z}^\const{e} \, \hat S_{\!z}^\const{p} - K_\const{e} \, B \, \hat S_{\!z}^\const{e} - K_\const{p} \, B \, \hat S_{\!z}^\const{p}\right) \ket{+-} \\
    &= A \, \frac{1}{2} \! \left( 0 \;+\; \ket{-+} \right) + \left(A \, \tfrac{1}{2} \left( -\tfrac{1}{2} \right) - K_\const{e} \, B \, \tfrac{1}{2} - K_\const{p} \, B \left( -\tfrac{1}{2} \right) \right)\ket{+-} \\
    &= \frac{A}{2} \ket{-+} + \left( - \frac{A}{4} + \frac{B}{2} \left(-K_\const{e} + K_\const{p} \right)\right) \ket{+-}
    \\\\
    \hat H \ket{-+}
    &= \left(A \, \frac{1}{2} \! \left( \hat S_{\!+}^\const{e} \hat S_{\!-}^\const{p} + \hat S_{\!-}^\const{e} \hat S_{\!+}^\const{p} \right) + A \,\hat S_{\!z}^\const{e} \, \hat S_{\!z}^\const{p} - K_\const{e} \, B \, \hat S_{\!z}^\const{e} - K_\const{p} \, B \, \hat S_{\!z}^\const{p}\right) \ket{-+} \\
    &= A \, \frac{1}{2} \! \left( \ket{+-} \;+\; 0 \right) + \left(A \left( -\tfrac{1}{2} \right) \tfrac{1}{2} - K_\const{e} \, B \left( -\tfrac{1}{2} \right) - K_\const{p} \, B \, \tfrac{1}{2} \right)\ket{-+} \\
    &= \frac{A}{2} \ket{+-} + \left( - \frac{A}{4} + \frac{B}{2} \left(K_\const{e} - K_\const{p} \right)\right) \ket{-+}
\end{align*}
Vidíme, že je hamiltonián částečně diagonalizovaný, stavy $\ket{++}$ a $\ket{--}$ jsou jeho vlastní stavy. Nalezneme ještě vlastní energie odpovídající superpozicím stavů $\ket{+-}$ a $\ket{-+}$:
\begin{gather*}
    \bra{\pm\mp} \hat H \ket{\pm\mp}
    = \mat{
        -\dfrac{A}{4} + \dfrac{B}{2}\left(-K_\const{e} + K_\const{p}\right) & \dfrac{A}{2} \\[15pt]
        \dfrac{A}{2} & -\dfrac{A}{4} - \dfrac{B}{2}\left(-K_\const{e} + K_\const{p}\right)
    } \eqqcolon M
    \: , \quad
    \alpha \coloneqq \frac{A}{4}
    \: , \quad
    \beta \coloneqq \frac{B}{2}\left(-K_\const{e} + K_\const{p}\right)
    \: .
    \\\\
    M = \mat{
        -\alpha + \beta & 2\alpha \\
        2\alpha & -\alpha - \beta
    }
    \: , \quad
    0 =
    \left| M - \lambda \1 \right| = \begin{vmatrix}
        -\alpha + \beta - \lambda & 2\alpha \\
        2\alpha & \!\!\! -\alpha - \beta - \lambda
    \end{vmatrix}
    = \lambda^2 + 2\alpha\lambda + \left(-3\alpha^2 - \beta^2\right)
    \\\\
    \lambda
    = \frac{-2\alpha \pm \sqrt{4\alpha^2 - 4\left(-3\alpha^2 - \beta^2\right)}}{2}
    = -\alpha \pm \sqrt{4\alpha^2 + \beta^2}
    = -\frac{A}{4} \pm \frac{1}{2} \sqrt{A^2 + B^2 \left(K_\const{e} - K_\const{p}\right)^2}
\end{gather*}
Vidíme tedy, že magnetické pole sejmulo degeneraci stacionárních stavů a máme 4 různé energetické hladiny:
\begin{table}[h!]
    \centering
    \begin{tabular}{ r|l }
        $E_0$ &
        $-\dfrac{A}{4} - \dfrac{1}{2} \sqrt{A^2 + B^2 \left(K_\const{e} - K_\const{p}\right)^2}$
        \\[10pt]\hline
        \rule{0pt}{1.5\normalbaselineskip}
        $E_1$ &
        $-\dfrac{A}{4} + \dfrac{1}{2} \sqrt{A^2 + B^2 \left(K_\const{e} - K_\const{p}\right)^2}$
        \\[10pt]\hline
        \rule{0pt}{1.5\normalbaselineskip}
        $E_2$ &
        $+\dfrac{A}{4} - \dfrac{B}{2} \left( K_\const{e} + K_\const{p}\right)$
        \\[10pt]\hline
        \rule{0pt}{1.5\normalbaselineskip}
        $E_3$ &
        $+\dfrac{A}{4} + \dfrac{B}{2} \left( K_\const{e} + K_\const{p}\right)$
    \end{tabular}
\end{table}

Pokračujeme spin-1 částicí. Zvolíme si bázi $\ket{+z}, \ket{0z}, \ket{-z}$, v ní vyjádříme $\vechat S$:
\begin{align*}
    \hat S_x &= \frac{1}{\oldsqrt{2}} \mat{
        0 & 1 & 0 \\
        1 & 0 & 1 \\
        0 & 1 & 0
    },
    &
    \hat S_y &= \frac{1}{\oldsqrt{2}} \mat{
        0 & -\i & 0 \\
        \i & 0 & -\i \\
        0 & \i & 0
    },
    &
    \hat S_z &= \mat{
        1 & 0 & 0 \\
        0 & 0 & 0 \\
        0 & 0 & -1
    },
    &
    \vechat S &= \mat{
        \hat S_x \\
        \hat S_y \\
        \hat S_z
    }.
\end{align*}
Budeme měřit spin v obecném směru $\vec n$, který si parametrizujeme sférickými souřadnicemi:
\begin{equation*}
    \vec n = \mat{
        \sin \vartheta \cos \varphi \\
        \sin \vartheta \sin \varphi \\
        \cos \vartheta
    }.
\end{equation*}
Zajímá nás, s jakou pravděpodobností naměříme stav
\begin{gather*}
    \left(\vechat S \cdot \vec n\right) \ket{0\vec n} = 0 \ket{0\vec n}
    \;\iff\;
    \ker \vechat S \cdot \vec n = \{\, \ket{0 \vec n} \,\}
    \\\\
    \hat S \cdot \vec n
    = \mat{
        \cos \vartheta &
        \dfrac{\sin \vartheta}{\oldsqrt{2}} \, \e{-\i \varphi} &
        0
        \\[15pt]
        \dfrac{\sin \vartheta}{\oldsqrt{2}} \, \e{+\i \varphi} \!\!&
        0 & \!\!
        \dfrac{\sin \vartheta}{\oldsqrt{2}} \, \e{-\i \varphi}
        \\[15pt]
        0 &
        \dfrac{\sin \vartheta}{\oldsqrt{2}} \, \e{+\i \varphi} &
        -\cos \vartheta
    }
    \sim \mat{
        \oldsqrt{2} \ctg \vartheta \, \e{+\i \varphi} \ph{+}&
        1  &
        0
        \\[15pt]
        \e{+\i \varphi} &
        0 &
        \e{-\i \varphi}
        \\[15pt]
        0 &
        1 &
        \ph{+}-\oldsqrt{2} \ctg \vartheta \, \e{-\i \varphi}
    }
    \to \mat{
        -\e{-\i \varphi} \\[15pt]
        \oldsqrt{2} \ctg \vartheta \\[15pt]
        \e{+\i \varphi}
    }
    \\\\
    \sqrt{
        |-\e{-\i \varphi}|^2 +
        |\oldsqrt{2} \ctg \vartheta|^2 +
        |\e{+\i \varphi}|^2
    }
    = \sqrt{
        2 + 2 \ctg^2 \vartheta
    }
    = \sqrt{2} \sqrt{1 + \frac{\cot^2 \vartheta}{\sin^2 \vartheta}}
    = \frac{\oldsqrt{2}}{\sin \vartheta}
    \\\\
    \ket{0\vec n}
    =
    \frac{1}{\; \dfrac{\oldsqrt{2}}{\sin \vartheta} \;}
    \mat{
        -\e{-\i \varphi} \\[7pt]
        \oldsqrt{2} \ctg \vartheta \\[7pt]
        \e{+\i \varphi}
    }
    =
    \frac{1}{\oldsqrt 2}
    \mat{
        -\e{-\i \varphi} \sin \vartheta \\[7pt]
        \oldsqrt{2} \cos \vartheta \\[7pt]
        \e{+\i \varphi} \sin \vartheta
    }
\end{gather*}
\begin{gather*}
    \left| \braket{+z}{0\vec n} \right|^2
    = \left|
    \frac{1}{\oldsqrt 2} \; \left( -\e{-\i \varphi} \sin \vartheta \right) \right|^2
    = \frac{1}{2} \sin^2 \vartheta
    \: .
\end{gather*}
Projekci do obecného směru tedy máme spočtenou. Vypočítáme ještě výsledek vylepšeného Stern-Gerlachova experimentu:
\begin{gather*}
    \hat P_{\pm x}
    = \ket{x+}\!\bra{x+} \;+\; \ket{x-}\!\bra{x-}
    = \mat{
        \nicefrac{1}{2} \\
        \nicefrac{1}{\oldsqrt 2} \\
        \nicefrac{1}{2}
    } \mat{
        \nicefrac{1}{2} &
        \nicefrac{1}{\oldsqrt 2} &
        \nicefrac{1}{2}
    }
    + \mat{
        \nicefrac{1}{2} \\
        \nicefrac{-1}{\oldsqrt 2} \\
        \nicefrac{1}{2}
    } \mat{
        \nicefrac{1}{2} &
        \nicefrac{-1}{\oldsqrt 2} &
        \nicefrac{1}{2}
    } = \mat{
        \nicefrac{1}{2} & 0 & \nicefrac{1}{2} \\
        0 & 1 & 0 \\
        \nicefrac{1}{2} & 0 & \nicefrac{1}{2}
    }
    \\
    \left| \bra{0z} \hat P_{\pm x} \ket{0z} \right|^2
    =
    \left|
        \mat{0 & 1 & 0}
        \mat{
            \nicefrac{1}{2} & 0 & \nicefrac{1}{2} \\
            0 & 1 & 0 \\
            \nicefrac{1}{2} & 0 & \nicefrac{1}{2}
        }
        \mat{0 \\ 1 \\ 0}
    \right|^2
    = 1
    \: .
\end{gather*}
Nakonec nalezneme energie stacionárních stavů pozitronia. Budeme opět pracovat v součinové bázi
\begin{equation*}
    \ket{\pm\pm} \coloneqq \ket{\pm z_{\const{e}^+}} \otimes \ket{\pm z_{\const{e}^-}} \: ,
\end{equation*}
a opět si vyjádříme polární  rozklad hamiltoniánu. Abychom se v neutopili v indexech, přejmenujeme si operátory spinů elektronu a pozitronu na $\vechat X, \vechat Y$.
\begin{equation*}
    \vechat X \coloneqq \vechat S^{\const{e}^-},
    \quad
    \vechat Y \coloneqq \vechat S^{\const{e}^+}
\end{equation*}
\begin{align*}
    \hat H
    &= A \, \vechat X \!\cdot \vechat Y
    + B \hat S^2
    =
    A \, \vechat X \!\cdot \vechat Y
    + B \, \big(\vechat X + \vechat Y\big) \cdot \big(\vechat X + \vechat Y\big)
    = B (\hat X^2 + \hat Y^2)
    + (A + 2B) \, \vechat X \! \cdot \vechat Y
    - B \big[ \vechat X \:{;}\: \vechat Y \big]
    \\[5pt]
    &= B \big(
        \frac{1}{2} (\hat X_+ \hat X_- + \hat X_- \hat X_+ + \hat Y_+ \hat Y_- + \hat Y_- \hat Y_+)
        + {\hat X_z}^2 + {\hat Y_z}^2
    \big)
    + (A + 2B)\big(
        \frac{1}{2} (\hat X_+ \hat Y_- + \hat X_- \hat Y_+) + \hat X_z \hat Y_z
    \big)
    \\[5pt]
    &= \frac{B}{2} \big(
        \hat X_+ \hat X_- + \hat X_- \hat X_+ + \hat Y_+ \hat Y_- + \hat Y_- \hat Y_+
        + 2 {\hat X_z}^2 + 2 {\hat Y_z}^2
    \big)
    + \big( \frac{A}{2} + B \big)\big(
        \hat X_+ \hat Y_- + \hat X_- \hat Y_+
        + 2 \hat X_z \hat Y_z
    \big)
    \: ,
\end{align*}
kde $[ \vechat A \:{;}\: \vechat B ]$ je komutátor skalárního součinu
\begin{gather*}
    \big[ \vechat A \:{;}\: \vechat B \big]
    = \vechat A \cdot \vechat B - \vechat B \cdot \vechat A
    = \comm{\hat A_x}{\hat B_x} + \comm{\hat A_y}{\hat B_y} + \comm{\hat A_z}{\hat B_z}
    \: ,
\end{gather*}
a pro $\vechat X, \vechat Y$ je nulový, protože operují na jiných částech součinového prostoru.

Pokračujeme výpočtem působení $\hat H$ na bázové stavy.
\begin{align*}
    \hat H \ket{++}
    &= \left(
        \frac{B}{2} \big(
            \hat X_+ \hat X_- + \hat X_- \hat X_+ + \hat Y_+ \hat Y_- + \hat Y_- \hat Y_+
            + 2 {\hat X_z}^2 + 2 {\hat Y_z}^2
        \big)
        + \big( \frac{A}{2} + B \big)\big(
            \hat X_+ \hat Y_- + \hat X_- \hat Y_+
            + 2 \hat X_z \hat Y_z
        \big)
    \right) \ket{++}
    \\[5pt]
    &= \left(
        \frac{B}{2} \big(
            1 + 0 + 1 + 0
            + 2 \left(\nicefrac{1}{2}\right)^2 + 2 \left(\nicefrac{1}{2}\right)^2
        \big)
        + \big( \frac{A}{2} + B \big)\big(
            0 + 0
            + 2 \left(\nicefrac{1}{2}\right) \left(\nicefrac{1}{2}\right)
        \big)
    \right) \ket{++}
    = \left( \frac{A}{4} + 2B \right) \ket{++}
    \\\\
    \hat H \ket{--}
    &= \left(
        \frac{B}{2} \big(
            \hat X_+ \hat X_- + \hat X_- \hat X_+ + \hat Y_+ \hat Y_- + \hat Y_- \hat Y_+
            + 2 {\hat X_z}^2 + 2 {\hat Y_z}^2
        \big)
        + \big( \frac{A}{2} + B \big)\big(
            \hat X_+ \hat Y_- + \hat X_- \hat Y_+
            + 2 \hat X_z \hat Y_z
        \big)
    \right) \ket{--}
    \\[5pt]
    &= \left(
        \frac{B}{2} \big(
            0 + 1 + 0 + 1
            + 2 \left(\nicefrac{-1}{2}\right)^2 + 2 \left(\nicefrac{-1}{2}\right)^2
        \big)
        + \big( \frac{A}{2} + B \big)\big(
            0 + 0
            + 2 \left(\nicefrac{-1}{2}\right) \left(\nicefrac{-1}{2}\right)
        \big)
    \right) \ket{--}
    = \left( \frac{A}{4} + 2B \right) \ket{--}
    \\\\
    \hat H \ket{+-}
    &= \left(
        \frac{B}{2} \big(
            \hat X_+ \hat X_- + \hat X_- \hat X_+ + \hat Y_+ \hat Y_- + \hat Y_- \hat Y_+
            + 2 {\hat X_z}^2 + 2 {\hat Y_z}^2
        \big)
        + \big( \frac{A}{2} + B \big)\big(
            \hat X_+ \hat Y_- + \hat X_- \hat Y_+
            + 2 \hat X_z \hat Y_z
        \big)
    \right) \ket{+-}
    \\[5pt]
    &=
    \frac{B}{2} \big(
        1 + 0 + 0 + 1
        + 2 \left(\nicefrac{1}{2}\right)^2 + 2 \left(\nicefrac{-1}{2}\right)^2
    \big)
    \ket{+-}
    +
    \big( \frac{A}{2} + B \big)\big(
        0 + \ket{-+}
        + 2 \left(\nicefrac{1}{2}\right) \left(\nicefrac{-1}{2}\right) \ket{+-}
    \big)
    \\[5pt]
    &= \big( -\frac{A}{4} + B \big) \ket{+-}
    + \big( \frac{A}{2} + B \big) \ket{-+}
    \\\\
    \hat H \ket{-+}
    &= \left(
        \frac{B}{2} \big(
            \hat X_+ \hat X_- + \hat X_- \hat X_+ + \hat Y_+ \hat Y_- + \hat Y_- \hat Y_+
            + 2 {\hat X_z}^2 + 2 {\hat Y_z}^2
        \big)
        + \big( \frac{A}{2} + B \big)\big(
            \hat X_+ \hat Y_- + \hat X_- \hat Y_+
            + 2 \hat X_z \hat Y_z
        \big)
    \right) \ket{-+}
    \\[5pt]
    &=
    \frac{B}{2} \big(
        0 + 1 + 1 + 0
        + 2 \left(\nicefrac{-1}{2}\right)^2 + 2 \left(\nicefrac{1}{2}\right)^2
    \big)
    \ket{-+}
    +
    \big( \frac{A}{2} + B \big)\big(
        \ket{+-} + 0
        + 2 \left(\nicefrac{-1}{2}\right) \left(\nicefrac{1}{2}\right) \ket{-+}
    \big)
    \\[5pt]
    &= \big( -\frac{A}{4} + B \big) \ket{-+}
    + \big( \frac{A}{2} + B \big) \ket{+-}
\end{align*}
Vidíme tedy, že $\ket{++}, \ket{--}$ jsou vlastní stavy degenerované energetické hladiny $E = \frac{A}{4} + 2B$. Dopočítáme vlastní energie v podprostoru $\ket{\pm\mp}$:
\begin{gather*}
    \bra{\pm\mp} \hat H \ket{\pm\mp}
    = \mat{
        -\dfrac{A}{4} + B & \ph{-}\dfrac{A}{2} + B \\[15pt]
        \ph{-}\dfrac{A}{2} + B & -\dfrac{A}{4} + B
    }
    \eqqcolon M
    \: , \quad
    0 = \left| M - \lambda \1 \right|^2
    = \begin{vmatrix}
        \; -\dfrac{A}{4} + B - \lambda & \dfrac{A}{2} + B \; \\[15pt]
        \; \dfrac{A}{2} + B & -\dfrac{A}{4} + B - \lambda \;
    \end{vmatrix}
    \\\\
    0 = \big(-\frac{A}{4} + B - \lambda \big)^2 - \big( \frac{A}{2} + B \big)^2
    \;\iff\;
    \lambda = -\frac{A}{4} + B \pm \big( \frac{A}{2} + B \big)
    \;\iff\;
    \lambda \in \big\{ -\frac{3A}{4}, \; \frac{A}{4} + 2B \big\}
\end{gather*}
Vidíme tedy, že pozitronium má jednu nedegenerovanou energetickou hladinu $E_0 = -\nicefrac{3}{4} \, A$ a jednu trojnásobně degenerovanou hladinu $E_1 = \nicefrac{A}{4} + 2B$. Tyto dvě hladiny odpovídají stavům s celkovým spinem $S=0$, resp. $S=1$.

\section{Jemné štěpení (spin-orbitální interakce)}

\subsection{Zadání}
Máme atom vodíku v p-stavu, jeho hamiltonián je
\begin{equation*}
    \hat H = A \, \vechat S \cdot \vechat L \: .
\end{equation*}
Spočtěte energie stacionárních stavů. Ukažte, že operátor $\vechat J = \vechat S + \vechat L$ komutuje s hamiltoniánem a vypočítejte působení $\hat J^2$ a $\hat J_z$ na stacionární stavy.

Tento atom následně vložíme do magnetického pole. Výsledný hamiltonián je
\begin{equation*}
    \hat H
    = A \, \vechat S \cdot \vechat L
    - K \big( \vechat S + \frac{1}{2} \vechat L \big) \cdot \vec B
    \: , \quad
    \vec B = (0, 0, B)
    \: ,
\end{equation*}
jaké jsou energie stacionárních stavů?

\subsection{Řešení}
V p-stavu může $L_z$ nabývat hodnot $-1$, $0$, $+1$, chová se tedy v podstatě jako spin-1. Budeme pracovat se součinovou bází vlastních stavů operátorů $\hat S_z, \hat L_z$:
\begin{align*}
    \ket{\pm\pm} &\coloneqq
    \ket{\pm_S} \otimes \ket{1, \pm 1_L}
    \\
    \ket{\pm\0} &\coloneqq
    \ket{\pm_S} \otimes \ket{1, \ph{\pm} 0_L}
\end{align*}

Opět si vyjádříme hamiltonián v polární formě a vypočteme jeho působení na $\ket{\pm \mask{+}{\ell}}$:
\begin{equation*}
    \hat H
    = \frac{A}{2} \left(
        \hat S_+ \hat L_- +
        \hat S_- \hat L_+
    \right)
    + A \, \hat S_z \hat L_z
\end{equation*}
\begin{align*}
    \hat H \ket{++}
    &= \left(
        \frac{A}{2} \left(
            \hat S_+ \hat L_- +
            \hat S_- \hat L_+
        \right)
        + A \, \hat S_z \hat L_z
    \right) \ket{++}
    = \left(
        \frac{A}{2} \left(
            0 \, \oldsqrt{2} +
            0
        \right)
        + A \left( +\nicefrac{1}{2} \right) (+1)
    \right) \ket{++}
    = \frac{A}{2} \ket{++}
    \\\\
    \hat H \ket{--}
    &= \left(
        \frac{A}{2} \left(
            \hat S_+ \hat L_- +
            \hat S_- \hat L_+
        \right)
        + A \, \hat S_z \hat L_z
    \right) \ket{--}
    = \left(
        \frac{A}{2} \left(
            0 +
            0 \, \oldsqrt{2}
        \right)
        + A \left( -\nicefrac{1}{2} \right) (-1)
    \right) \ket{--}
    = \frac{A}{2} \ket{--}
    \\\\
    \hat H \ket{+\0}
    &= \left(
        \frac{A}{2} \left(
            \hat S_+ \hat L_- +
            \hat S_- \hat L_+
        \right)
        + A \, \hat S_z \hat L_z
    \right) \ket{+\0}
    =
    \frac{A}{2} \left(
        0 +
        \oldsqrt{2} \ket{-+}
    \right)
    + A \left( \nicefrac{1}{2} \right) 0
    = \oldsqrt{2} \frac{A}{2} \ket{-+}
    = \frac{A}{\oldsqrt{2}} \ket{-+}
    \\\\
    \hat H \ket{-\0}
    &= \left(
        \frac{A}{2} \left(
            \hat S_+ \hat L_- +
            \hat S_- \hat L_+
        \right)
        + A \, \hat S_z \hat L_z
    \right) \ket{-\0}
    =
    \frac{A}{2} \left(
        \oldsqrt{2} \ket{+-}
        + 0
    \right)
    + A \left( \nicefrac{1}{2} \right) 0
    = \oldsqrt{2} \frac{A}{2} \ket{+-}
    = \frac{A}{\oldsqrt{2}} \ket{+-}
    \\\\
    \hat H \ket{+-}
    &= \left(
        \frac{A}{2} \left(
            \hat S_+ \hat L_- +
            \hat S_- \hat L_+
        \right)
        + A \, \hat S_z \hat L_z
    \right) \ket{+-}
    =
    \frac{A}{2} \left(
        0 +
        \oldsqrt{2} \ket{-\0}
    \right)
    + A \left( +\nicefrac{1}{2} \right) (-1) \ket{+-}
    = \frac{A}{\oldsqrt 2} \ket{-\0} - \frac{A}{2} \ket{+-}
    \\\\
    \hat H \ket{-+}
    &= \left(
        \frac{A}{2} \left(
            \hat S_+ \hat L_- +
            \hat S_- \hat L_+
        \right)
        + A \, \hat S_z \hat L_z
    \right) \ket{-+}
    =
    \frac{A}{2} \left(
        \oldsqrt{2} \ket{+\0}
        + 0
    \right)
    + A \left( -\nicefrac{1}{2} \right) (+1) \ket{+-}
    = \frac{A}{\oldsqrt 2} \ket{+\0} - \frac{A}{2} \ket{-+}
\end{align*}
Vidíme, že hamiltonián je blokově diagonální. Stavy $\ket{++}$ a $\ket{--}$ jsou stacionární stavy s energií $\nicefrac{A}{2}$. Podprostory $\Span\{ \ket{+\0}, \ket{-+} \}$ a $\Span\{ \ket{-\0}, \ket{+-} \}$ můžeme diagonalizovat samostatně.
\begin{gather*}
    M_+ \coloneqq \mat{
        \bra{+\0} \dfrac{2}{A} \hat H \ket{+\0} &
        \bra{+\0} \dfrac{2}{A} \hat H \ket{-+} \\[15pt]
        \bra{-+} \dfrac{2}{A} \hat H \ket{+\0} &
        \bra{-+} \dfrac{2}{A} \hat H \ket{-+}
    }
    = \mat{
        0 & \oldsqrt{2} \\[5pt]
        \oldsqrt{2} & -1
    }
\end{gather*}
Vlastní čísla matice $M$ jsou $-2$ a $1$, jim odpovídají vlastní vektory $(1 ,\, -\oldsqrt 2)$ a $(\oldsqrt 2 ,\, 1)$. Nové stacionární stavy si nazveme nenápaditě $\ket{A}, \ket{B}$.
\begin{gather*}
    \ket A \coloneqq \frac{1}{\oldsqrt 3} \big(
        \ket{+\0} - \oldsqrt 2 \ket{-+}
    \big)
    \: , \quad
    \ket B \coloneqq \frac{1}{\oldsqrt 3} \big(
        \oldsqrt 2 \ket{+\0} + \ket{-+}
    \big)
    \: ,
    \\\\
    \hat H \ket A
    = \frac{A}{2} (-2) \ket A
    = -A \ket A
    \: , \quad
    \hat H \ket B
    = \frac{A}{2} \ket B
    \: .
\end{gather*}
Obdobně pracujeme i se stavy $\ket{-\0}, \ket{+-}$.
\begin{gather*}
    M_- \coloneqq \mat{
        \bra{-\0} \dfrac{2}{A} \hat H \ket{-\0} &
        \bra{-\0} \dfrac{2}{A} \hat H \ket{+-} \\[15pt]
        \bra{+-} \dfrac{2}{A} \hat H \ket{-\0} &
        \bra{+-} \dfrac{2}{A} \hat H \ket{+-}
    }
    = \mat{
        0 & \oldsqrt{2} \\[5pt]
        \oldsqrt{2} & -1
    }
    \\\\
    \ket C \coloneqq \frac{1}{\oldsqrt 3} \big(
        \ket{-\0} - \oldsqrt 2 \ket{+-}
    \big)
    \: , \quad
    \ket D \coloneqq \frac{1}{\oldsqrt 3} \big(
        \oldsqrt 2 \ket{-\0} + \ket{+-}
    \big)
    \: ,
    \\\\
    \hat H \ket C
    = \frac{A}{2} (-2) \ket C
    = -A \ket C
    \: , \quad
    \hat H \ket D
    = \frac{A}{2} \ket D
    \: .
\end{gather*}
Vidíme, že hamiltonián má jednu čtyřikrát degenerovanou energetickou hladinu $\nicefrac{A}{2}$ a jednu dvakrát degenerovanou hladinu $-A$. Stacionární stavy tvoří bázi hilbertova prostoru, aby se nám s nimi lépe dále pracovalo, přejmenujeme~si~je:
\begin{align*}
    \ket{ \nicefrac{3}{2}, +\nicefrac{3}{2} }
    &\coloneqq \ket{++}
    \: , \\[10pt]
    \ket{ \nicefrac{3}{2}, +\nicefrac{1}{2} }
    &\coloneqq \ket{B}
    = \frac{1}{\oldsqrt 3} \big(
        \oldsqrt 2 \ket{+\0} + \ket{-+}
    \big)
    \: , \\[10pt]
    \ket{ \nicefrac{3}{2}, -\nicefrac{1}{2} }
    &\coloneqq \ket{D}
    = \frac{1}{\oldsqrt 3} \big(
        \oldsqrt 2 \ket{-\0} + \ket{+-}
    \big)
    \: , \\[10pt]
    \ket{ \nicefrac{3}{2}, -\nicefrac{3}{2} }
    &\coloneqq \ket{--}
    \: , \\[10pt]
    \ket{ \nicefrac{1}{2}, +\nicefrac{1}{2} }
    &\coloneqq \ket{A}
    = \frac{1}{\oldsqrt 3} \big(
        \ket{+\0} - \oldsqrt 2 \ket{-+}
    \big)
    \: , \\[10pt]
    \ket{ \nicefrac{1}{2}, -\nicefrac{1}{2} }
    &\coloneqq \ket{C}
    = \frac{1}{\oldsqrt 3} \big(
        \ket{-\0} - \oldsqrt 2 \ket{+-}
    \big)
    \: .
\end{align*}
Dále budeme pracovat s operátorem $\vechat J = \vechat S + \vechat L$. Nejprve ukážeme, že komutuje s hamiltoniánem.
\begin{gather*}
    \comm{\hat S_j}{\; \hat H}
    = A \, \comm{\hat S_j}{\; \vechat S \cdot \vechat L}
    = A \sum_k \comm{\hat S_j}{\; \hat S_k \hat L_k}
    = A \sum_k \comm{\hat S_j}{\; \hat S_k} \hat L_k
    + A \sum_k \hat S_k \comm{\hat S_j}{\; \hat L_k}
    = A \sum_k \i \, \varepsilon_{jk\ell} \, \hat S_\ell \hat L_k
    \\[10pt]
    \comm{\hat L_j}{\; \hat H}
    = A \, \comm{\hat L_j}{\; \vechat S \cdot \vechat L}
    = A \sum_k \comm{\hat L_j}{\; \hat S_k \hat L_k}
    = A \sum_k \comm{\hat L_j}{\; \hat S_k} \hat L_k
    + A \sum_k \hat S_k \comm{\hat L_j}{\; \hat L_k}
    = A \sum_k \i \, \varepsilon_{jk\ell} \, \hat S_k \hat L_\ell
    \\[10pt]
    \comm{\hat J_j}{\; \hat H}
    = \comm{\hat S_j}{\; \hat H}
    + \comm{\hat L_j}{\; \hat H}
    = A \, \sum_k \, \i \, \varepsilon_{jk\ell} \, \hat S_\ell \hat L_k
    \; + \;
    A \, \sum_k \, \i \, \varepsilon_{jk\ell} \, \hat S_k \hat L_\ell
    = A \, \sum_k \, \i \, \big(
        \underbrace{
            \varepsilon_{jk\ell} +
            \varepsilon_{j\ell k}
        }_0
    \big) \, \hat S_\ell \hat L_k
    = 0
    \\[10pt]
    \comm{\vechat J}{\; \hat H} = \big(
        \comm{\hat J_x}{\hat H}, \;
        \comm{\hat J_y}{\hat H}, \;
        \comm{\hat J_z}{\hat H}
    \big)
    = \big( 0 ,\; 0 ,\; 0\big)
\end{gather*}
Pokračujeme výpočtem působení $\hat J_z$ na $\ket{j,m}$.
\begin{gather*}
    \hat J_z \ket{ \nicefrac{3}{2}, +\nicefrac{3}{2} }
    = \big( \hat S_z + J_z \big) \ket{++}
    = \big( +\frac{1}{2} + 1 \big) \ket{++}
    = +\frac{3}{2} \ket{ \nicefrac{3}{2}, +\nicefrac{3}{2} }
    \\\\
    \hat J_z \ket{ \nicefrac{3}{2}, -\nicefrac{3}{2} }
    = \big( \hat S_z + J_z \big) \ket{--}
    = \big( -\frac{1}{2}  + 1 \big) \ket{--}
    = -\frac{3}{2} \ket{ \nicefrac{3}{2}, -\nicefrac{3}{2} }
    \\\\
    \hat J_z \ket{ \nicefrac{3}{2}, +\nicefrac{1}{2} }
    = \big( \hat S_z + J_z \big)
    \frac{1}{\oldsqrt 3} \big(
        \oldsqrt 2 \ket{+\0} + \ket{-+}
    \big)
    = \frac{1}{\oldsqrt 3} \big(
        \oldsqrt 2
        (+\frac{1}{2} + 0)
        \ket{+\0}
        +
        (-\frac{1}{2} + 1)
        \ket{-+}
    \big)
    =
    +\frac{1}{2}
    \ket{ \nicefrac{3}{2}, +\nicefrac{1}{2} }
    \\\\
    \hat J_z \ket{ \nicefrac{3}{2}, -\nicefrac{1}{2} }
    = \big( \hat S_z + J_z \big)
    \frac{1}{\oldsqrt 3} \big(
        \oldsqrt 2 \ket{-\0} + \ket{+-}
    \big)
    = \frac{1}{\oldsqrt 3} \big(
        \oldsqrt 2
        (-\frac{1}{2} + 0)
        \ket{-\0}
        +
        (+\frac{1}{2} - 1)
        \ket{+-}
    \big)
    =
    -\frac{1}{2}
    \ket{ \nicefrac{3}{2}, -\nicefrac{1}{2} }
    \\\\
    \hat J_z \ket{ \nicefrac{1}{2}, +\nicefrac{1}{2} }
    = \big( \hat S_z + J_z \big)
    \frac{1}{\oldsqrt 3} \big(
        \ket{+\0} - \oldsqrt 2 \ket{-+}
    \big)
    = \frac{1}{\oldsqrt 3} \big(
        (+\frac{1}{2} + 0)
        \ket{+\0}
        -
        \oldsqrt 2
        (-\frac{1}{2} + 1)
        \ket{-+}
    \big)
    =
    +\frac{1}{2}
    \ket{ \nicefrac{1}{2}, +\nicefrac{1}{2} }
    \\\\
    \hat J_z \ket{ \nicefrac{1}{2}, -\nicefrac{1}{2} }
    = \big( \hat S_z + J_z \big)
    \frac{1}{\oldsqrt 3} \big(
        \ket{-\0} - \oldsqrt 2 \ket{+-}
    \big)
    = \frac{1}{\oldsqrt 3} \big(
        (-\frac{1}{2} + 0)
        \ket{-\0}
        -
        \oldsqrt 2
        (+\frac{1}{2} - 1)
        \ket{+-}
    \big)
    =
    -\frac{1}{2}
    \ket{ \nicefrac{1}{2}, -\nicefrac{1}{2} }
\end{gather*}
Dále vypočítáme působení $\hat J^2$ na $\ket{j,m}$.
\begin{gather*}
    \hat J^2
    = \left( \vechat S + \vechat L \right)^2
    = \hat S^2 + \hat L^2 + 2 \, \vechat S \cdot \vechat L - \big[ \vechat S \,{;}\, \vechat L \big]
    = \hat S^2 + \hat L^2 + \frac{2}{A} \hat H
    \\\\
    \hat J^2
    = \frac{1}{2} \big(
        \hat S_+ \hat S_- +
        \hat S_- \hat S_+ +
        \hat L_+ \hat L_- +
        \hat L_- \hat L_+
    \big)
    + {\hat S_z}^2 + {\hat L_z}^2
    + \frac{2}{A} \hat H
\end{gather*}
Vynásobíme-li výraz stavem $\ket{j,m}$, stejným způsobem jako nahoře, vyjde nám:
\begin{equation*}
    \hat J^2 \ket{\nicefrac{3}{2}, m}
    = \nicefrac{3}{2} \; (\nicefrac{3}{2} + 1) \,
    \ket{\nicefrac{3}{2}, m}
    \: , \hspace{3em}
    \hat J^2 \ket{\nicefrac{1}{2}, m}
    = \nicefrac{1}{2} \; (\nicefrac{1}{2} + 1) \,
    \ket{\nicefrac{1}{2}, m}
    \: , \hspace{3em}
    \forall m = -\tfrac{3}{2}, \; {...}, \; \tfrac{3}{2}
    \: .
\end{equation*}


\end{document}
