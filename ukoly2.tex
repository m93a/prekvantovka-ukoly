% !TEX program = xelatex

\documentclass[10pt,a4paper]{article}
\usepackage[top = 1.5cm, bottom = 1.5cm, left = 1.5cm, right = 1.5cm]{geometry}

\usepackage{titling}
\usepackage[czech]{babel}
\usepackage{graphicx}
\usepackage{lmodern}
\usepackage{hyperref}
\usepackage{setspace}
\usepackage{csvsimple}

\usepackage{amsmath}
\usepackage{amssymb}
\usepackage{gensymb}
\usepackage{units}
\usepackage{bm}
\usepackage{bbm}
\delimitershortfall=-1pt

\usepackage{mathtools}
\usepackage{accents}
\usepackage{calc}

% no page break
\newenvironment{absolutelynopagebreak}
  {\par\nobreak\vfil\penalty0\vfilneg
   \vtop\bgroup}
  {\par\xdef\tpd{\the\prevdepth}\egroup
   \prevdepth=\tpd}


% redefine \sqrt
\usepackage{letltxmacro}
\makeatletter
\let\oldr@@t\r@@t
\def\r@@t#1#2{%
\setbox0=\hbox{$\oldr@@t#1{#2\,}$}\dimen0=\ht0
\advance\dimen0-0.2\ht0
\setbox2=\hbox{\vrule height\ht0 depth -\dimen0}%
{\box0\lower0.4pt\box2}}
\LetLtxMacro{\oldsqrt}{\sqrt}
\renewcommand*{\sqrt}[2][\ ]{\oldsqrt[#1]{#2\,}\,}
\makeatother

% redefine \hbar
\LetLtxMacro{\oldhbar}{\hbar}
\renewcommand*{\hbar}{{\mathpalette\hbaraux\relax\mathrm{h}}}
\newcommand*{\hbaraux}[2]{\sbox0{\mathsurround=0pt$#1\mathchar'26$}\mkern-1mu\lower.07\ht0\box0\mkern-8mu}

% define \bigcdot
\makeatletter
\newcommand*\bigcdot{\mathpalette\bigcdot@{.8}}
\newcommand*\bigcdot@[2]{\mathbin{\vcenter{\hbox{\scalebox{#2}{$\m@th#1\bullet$}}}}}
\makeatother

\def\ph{\phantom}
\def\vph{\vphantom}
\def\hph{\hphantom}
\def\rzw{\mathrlap}
\def\lzw{\mathllap}
\def\czw{\mathclap}

\newcommand{\nph}[1]{\settowidth{\dimen0}{#1}\hspace*{-\dimen0}}

\newcommand*{\mask}[2]{%
    \mathord{\makebox[\widthof{\(#1\)}]{\(#2\)}}%
}

\def\?{\mathit{?}}

\newcommand{\comm}[2]{\left[ #1, #2 \right]}
\newcommand{\const}[1]{\text{#1}}
\newcommand{\norm}[1]{\left\lVert#1\right\rVert}

\newcommand{\mat}[1]{
    \begin{pmatrix}
        #1
    \end{pmatrix}
}

\newcommand{\mata}[2]{
    \left(
    \begin{array}{@{}#1@{}}
        #2
    \end{array}
    \right)
}

\newcommand{\smat}[2][1]{
    \scalebox{#1}{$\mat{#2}$}
}

\renewcommand{\d}[1]{\;\const{d}#1}
\newcommand{\dd}[2]{\frac{\const{d} #1}{\const{d} #2} \;}
\newcommand{\pd}[2]{\frac{\partial  #1}{\partial  #2} \;}

\newcommand{\bra}[1]{\left< #1 \right|}
\newcommand{\ket}[1]{\left| #1 \right>}
\newcommand{\braket}[2]{\left< #1 \middle| #2 \right>}

\newcommand{\e}[1]{\const{e}^{#1}}
\renewcommand{\i}{\const{i}}

\newcommand{\bigdot}[1]{\accentset{\bigcdot}{#1}}
\newcommand{\bigddot}[1]{\accentset{\bigcdot\bigcdot}{#1}}


\def\kzero{\ket{\mask{+}{0}}}
\def\kone{\ket{\mask{+}{1}}}
\def\ktwo{\ket{\mask{+}{2}}}
\def\kplus{\ket{+}}
\def\kminus{\ket{-}}

\def\bone{\bra{\mask{+}{1}}}
\def\btwo{\bra{\mask{+}{2}}}

\def\R{\mathbb{R}}
\def\C{\mathbb{C}}
\def\1{\mathbbm{1}}

\def\Parity{\hat{\mathcal P}}
\def\Cop{\hat{\mathcal C}}
\def\G{\mathnormal\Gamma}


\begin{document}

\title{Úvod do kvantové mechaniky: Domácí úkoly z přednášek}
\author{Michal Grňo}
\date{\today}

\maketitle

\section{Projekce spinu do obecného směru}

\subsection{Zadání}
Nechť projekce spinu do osy $z$ je $\nicefrac{1}{2}$. S jakou pravděpodobností naměříme projekci spinu $\pm\nicefrac{1}{2}$ do obecného směru?

\subsection{Řešení}
Zavedeme si jednotkový vektor $\vec n$ parametrizovaný sférickými souřadnicemi:
\begin{equation*}
    \vec n = \mat{
        \sin \vartheta \cos \varphi \\
        \sin \vartheta \sin \varphi \\
        \cos \vartheta
    }
\end{equation*}
Zavedeme si operátor $\hat S_{\vec n} = \vec n \cdot \hat{\vec S}$, kde $\hat{\vec S}$ reprezentujeme Pauliho maticemi:
\begin{equation*}
    \hat{\vec S} = \frac{1}{2} \mat{
        \smat[0.8]{0 & \ph{-} 1 \\ 1 & \ph{-} 0} \\[10pt]
        \smat[0.8]{0 & -\i \\ \i & \ph{-} 0} \\[10pt]
        \smat[0.8]{1 & \ph{-} 0 \\ 0 & -1}
    }
\end{equation*}
Operátor $\hat S_{\vec n}$ nám potom vyjde:
\begin{equation*}
    \hat S_{\vec n} = \frac{1}{2} \mat{
        \cos \vartheta & \sin \vartheta \; \e{-\i \varphi} \\
        \sin \vartheta \; \e{\i \varphi} & -\cos \vartheta
    }
\end{equation*}
Víme, že vlastní čísla $\hat S_{\vec n}$ jsou $\pm \nicefrac{1}{2}$, přejdeme tedy rovnou k nalezení vlastních vektorů:
\begin{equation*}
    \ker(\hat S_{\vec n} - \nicefrac{1}{2} \, \hat I)
    = \ker\mat{
        \cos \vartheta - 1 & \sin \vartheta \; \e{-\i \varphi} \\
        \sin \vartheta \; \e{\i \varphi} & -\cos \vartheta - 1
    }
    = \operatorname{span} \{ \mat{
        \e{-\i \varphi} (\cot \vartheta + \csc \vartheta) \\ 1
    } \}
\end{equation*}
\begin{equation*}
    \ker(\hat S_{\vec n} + \nicefrac{1}{2} \, \hat I)
    = \ker\mat{
        \cos \vartheta + 1 & \sin \vartheta \; \e{-\i \varphi} \\
        \sin \vartheta \; \e{\i \varphi} & -\cos \vartheta + 1
    }
    = \operatorname{span} \{ \mat{
        \e{-\i \varphi} (\cot \vartheta - \csc \vartheta) \\ 1
    } \}
\end{equation*}
Normalizované vlastní stavy jsou tedy:
\begin{equation*}
    \ket{\pm \vec n} =
    \frac{1}{\oldsqrt{2}} \,
    \frac{1}{\sqrt{1 \pm \cos \vartheta}}
    \mat{
        \cos \vartheta \pm 1 \\
        \e{\i \varphi} \sin \vartheta
    }
\end{equation*}
Pravděpodobnost naměření $\ket{\pm \vec n}$, je-li stav $\ket{+z}$, je:
\begin{equation*}
    P = | \braket{+z}{\pm \vec n} |^2
    =
    \left|
    \mat{1 \\ 0}
    \cdot
    \frac{1}{\oldsqrt{2}} \,
    \frac{1}{\sqrt{1 \pm \cos \vartheta}}
    \mat{
        \cos \vartheta \pm 1 \\
        \e{\i \varphi} \sin \vartheta
    }
    \right|^2
    =
    \left|
    \frac{1}{\oldsqrt{2}} \,
    \frac{\cos \vartheta \pm 1}{\sqrt{1 \pm \cos \vartheta}}
    \right|^2
    = \frac{1}{2} \pm \frac{1}{2} \cos \theta.
\end{equation*}


\section{Rabiho metoda}

\subsection{Zadání}
Mějme částici se spinem $\nicefrac{1}{2}$ v poli s intenzitou
\begin{equation*}
    \vec B = \mat{
        B_1 \cos \omega t \\
        B_1 \sin \omega t \\
        B_0
    },
\end{equation*}
kde $B_1 \ll B_0$, $\omega \approx -K B_0$.

Stav spinu $\ket{\psi(t)}$ začíná v čase $t=0$ jako $\ket{\pm z}$. S jakou pravděpodobností bude v obecném čase $t$ ve stavu $\ket{-z}$?

\subsection{Řešení}
Hamiltonián systému je
\begin{equation*}
    \hat H = - K \; \hat{\vec S} \cdot \vec B,
\end{equation*}
kde $\hat{\vec S}$ reprezentujeme Pauliho maticemi. Využijeme rozklad $\hat{\vec S}$ na žebříkové operátory $\hat S_\pm$:
\begin{equation*}
    \hat S_\pm
    = \hat S_\mathrm{x} \pm \i \hat S_\mathrm{y}
    = \frac{1}{2} \mat{ 0 & 1 \pm 1 \\ 1 \mp 1 & 0 }
    = \begin{Bmatrix}
        \smat[0.8]{0 & 1 \\ 0 & 0} \\[15pt]
        \smat[0.8]{0 & 0 \\ 1 & 0}
    \end{Bmatrix}
    = \ket{\pm} \bra{\mp}.
\end{equation*}
Navíc víme, že
\begin{equation*}
    \hat S_\mathrm{z}
    = \frac{1}{2} (\ket{+} \! \bra{+} \, - \, \ket{-} \! \bra{-}).
\end{equation*}
Podobně rozložíme $\vec B$:
\begin{equation*}
    B_\pm
    = B_\mathrm{x} \pm \i B_\mathrm{y}
    = B_1 (\cos \omega t \pm \i \sin \omega t)
    = B_1 \, \e{\pm \i \omega t}.
\end{equation*}
Nyní můžeme vyjádřit hamiltonián ve tvaru
\begin{equation*}
    \hat H
    = -K \; \hat{\vec S} \cdot \vec B
    = -K \; \left(
        \frac{1}{2} (\hat S_+ B_- + \hat S_- B_+)
        + \hat S_\mathrm{z} B_\mathrm{z}
    \right)
    = -\frac{K}{2} \left(
        B_1 \, \e{-\i \omega t} \ket{+} \! \bra{-} \, + \,
        B_1 \, \e{+\i \omega t} \ket{-} \! \bra{+} \, + \,
        \ket{+} \! \bra{+} \, - \, \ket{-} \! \bra{-}
    \right),
\end{equation*}
tedy v maticové formě
\begin{equation*}
    \bra{\pm} \hat H \ket{\pm} =
    -\frac{K}{2} \mat{
        B_0 & B_1 \e{-\i \omega t}  \\
        B_1 \e{+\i \omega t} & -B_0
    }.
\end{equation*}

\bigskip

Nyní se můžeme pustit do řešení samotné Schrödingerovy rovnice.
\begin{equation*}
    -\i \dd{}{t}\!\! \ket{\psi} = \hat H(t) \ket{\psi}
\end{equation*}
\begin{equation*}
    -\i \dd{}{t} \mat{ c_+(t) \\ c_-(t) }
    = -\frac{K}{2} \mat{
        B_0 & B_1 \e{-\i \omega t}  \\
        B_1 \e{+\i \omega t} & -B_0
    }
    \mat { c_+(t) \\ c_-(t) }
\end{equation*}
\begin{align}
    -\i\bigdot c_+
    &= - \frac{K B_0}{2} \; \, c_+
    - \frac{K B_1}{2} \; \, \e{-\i \omega t} \, c_-
    \label{rce_cplus}
    \\
    -\i\bigdot c_-
    &= + \frac{K B_0}{2} \; \, c_-
    - \frac{K B_1}{2} \; \, \e{+\i \omega t} \, c_+
    \label{rce_cminus}
\end{align}
\begin{gather*}
    \text{Z rovnice \eqref{rce_cplus}: }
    \quad
    c_- = \frac{2}{K B_1} \e{+\i \omega t} \left( \i \bigdot c_+ - \frac{K B_0}{2} \, c_+ \right)
    = \e{+\i \omega t} \left(
        \i \; \frac{2}{K B_1} \, \bigdot{c}_+
        - \frac{B_0}{B_1} \, c_+
    \right)
    \\
\end{gather*}
\begin{gather*}
    \text{Z rovnice \eqref{rce_cminus}: }
    \\
    -\i \dd{}{t}
    \e{+\i \omega t}
    \left(
        \i \; \frac{2}{K B_1} \, \bigdot{c}_+
        - \frac{B_0}{B_1} \, c_+
    \right)
    =
    \frac{K B_0}{2} \; \, \e{+\i \omega t}
    \left(
        \i \; \frac{2}{K B_1} \, \bigdot{c}_+
        - \frac{B_0}{B_1} \, c_+
    \right)
    - \frac{K B_1}{2} \; \, \e{+\i \omega t} \, c_+
    \\[10pt]
    \big\Downarrow
    \\[10pt]
    0 = \bigddot c_+ + \i\omega \bigdot c_+ +
    \underbrace{\left(
        \frac{ {B_0}^2 K^2 }{4}
        - \frac{ B_0 K \omega }{2}
        - \frac{ {B_1}^2 K^2 }{4}
    \right)}_\kappa c_+
\end{gather*}
Máme tedy rovnici typu
\begin{align*}
    f'' + \i\omega f' + \kappa f &= 0 \\
    \lambda^2 + \i\omega\rzw{\lambda}\ph{f'} + \kappa\ph{f} &= 0
\end{align*}
\begin{gather*}
    \lambda
    = \frac{-\i \omega \pm \sqrt{(\i\omega)^2 - 4 \kappa}}{2}
    = - \frac{\i}{2} \, \omega \pm \frac{\i}{2} \sqrt{\omega^2 + 4\kappa}
    \\[10pt]
    f
    = C_1 \, \exp\i(-\frac{\omega}{2} + \frac{1}{2} \sqrt{\omega^2 + 4\kappa}) t
    + C_2 \, \exp\i(-\frac{\omega}{2} - \frac{1}{2} \sqrt{\omega^2 + 4\kappa}) t
\end{gather*}
Odmocninu můžeme ještě dále zjednodušit zanedbáním členu s ${B_1}^2$, který je výrazně menší než ostatní členy (viz zadání).
\begin{gather*}
    \sqrt{\omega^2 + 4\kappa}
    = \sqrt{
        \omega^2
        + {B_0}^2 K^2
        - 2 B_0 K \omega
        - {B_1}^2 K^2
    } \approx  \sqrt{
        \omega^2
        - 2 B_0 K \omega
        + {B_0}^2 K^2
    }
    = B_0 K - \omega
\end{gather*}
Pro $c_+$ tedy dostáváme:
\begin{align*}
    c_+(t) &= \e{-\i\omega t/2} \left(
        C_1 \e{+\i t\, \frac{B_0 K - \omega}{2}} +
        C_2 \e{-\i t\, \frac{B_0 K - \omega}{2}}
    \right)
    \\[10pt]
    c_+(t) &= \e{-\i\omega t/2} \left(
        D_1 \cos \frac{B_0 K - \omega}{2} t +
        D_2 \sin \frac{B_0 K - \omega}{2} t
    \right)
\end{align*}
Dosazením do \eqref{rce_cminus} získáme:
\begin{gather*}
    c_-(t) = \e{+\i \omega t} \e{-\omega t/2} \left(
        \i \; \frac{2}{K B_1} \, \dd{}{t} \left(
            D_1 \cos \tfrac{B_0 K - \omega}{2} t +
            D_2 \sin \tfrac{B_0 K - \omega}{2} t
        \right)
        - \frac{B_0}{B_1} \, \left(
            D_1 \cos \tfrac{B_0 K - \omega}{2} t +
            D_2 \sin \tfrac{B_0 K - \omega}{2} t
        \right)
    \right)
\end{gather*}
Konstanty $D_n$ určíme z počáteční podmínky $\ket{\psi(t=0)} = \ket{\pm z}$ a z požadavku, aby byl stav $\ket{\psi(t)}$ normalizovaný. Pro přehlednost si zavedeme označení $\psi_\pm(0) \equiv \ket{\pm z}$.
\begin{gather*}
    1 = \braket{\pm z}{\psi_\pm(0)}
    = \begin{Bmatrix}
        \smat[0.8]{1 \\ 0} \\[15pt]
        \smat[0.8]{0 \\ 1}
    \end{Bmatrix}
    \cdot \mat{
        D_1 \\ D_3
    }
\end{gather*}
Tedy pro $\psi_+$ máme $D_1 = 1, \; D_2 = 0$, pro $\psi_-$ zase $D_3 = 1, \; D_4 = 0$. Zbylé konstanty dopočítáme dosazením. Celkově platí:
\begin{equation*}
    \ket{\psi_+(t)} = \mat{
        \ph{\i \,} \e{\displaystyle -\i\omega t/2} \;\; \cos \dfrac{B_0 K - \omega}{2} t
        \\[10pt]
        \i \, \e{\displaystyle +\i\omega t/2} \;\; \sin \dfrac{B_0 K - \omega}{2} t
    } \: ,
    \quad
    \ket{\psi_-(t)} = \mat{
        \i \, \e{\displaystyle -\i\omega t/2} \;\; \sin \dfrac{B_0 K - \omega}{2} t
        \\[10pt]
        \ph{\i \,} \e{\displaystyle +\i\omega t/2} \;\; \cos \dfrac{B_0 K - \omega}{2} t
    } \: .
\end{equation*}
Pokud by nastal případ $\omega = -B_0 K$, máme:
\begin{equation*}
    \ket{\psi_+(t)} = \mat{
        \ph{\i \,} \e{\displaystyle +\i K B_0 t/2} \;\; \cos B_0 K t
        \\[10pt]
        \i \, \e{\displaystyle -\i K B_0 t/2} \;\; \sin B_0 K t
    } \: ,
    \quad
    \ket{\psi_-(t)} = \mat{
        \i \, \e{\displaystyle +\i K B_0 t/2} \;\; \sin B_0 K t
        \\[10pt]
        \ph{\i \,} \e{\displaystyle -\i K B_0 t/2} \;\; \cos B_0 K t
    } \: .
\end{equation*}



\section{\texorpdfstring{$\mathcal{PT}$}{PT}-symetrický hamiltonián}
\subsection{Zadání}
Máme zadány operátory
\begin{equation*}
    \hat\G = \smat[0.8]{
        \ph{-}K & -\i a \\
        - \i a  & -K
    },
    \quad
    \Parity = \smat[0.8]{
        1 & \ph{-}0 \\
        0 & -1
    },
\end{equation*}
kde $K, a \in \R_+$. Ukažte, že
\begin{enumerate}
    \item $\Parity^2 = \1, \quad \hat\G^+ = \Parity \; \hat\G \; \Parity$
    \item $\hat\G \ket{\pm} = \pm \G \ket{\pm}, \quad \G = \sqrt{K^2 - a^2}, \quad$ tedy pro $a<K$ je neporušena $\mathcal{PT}$-symetrie.
    \item pro $a>K$ platí $\bra{\pm} \Parity \ket{\pm} = 0$
\end{enumerate}

Pro $\mathcal{PT}$-symetrické operátory platí upravená relace úplnosti
\begin{equation*}
    \1 = \sum_n (-1)^n \ket{n} \bra{n} \Parity .
\end{equation*}
Ověřte její platnost pro $\hat\G$ při $a<K$.

Definujeme operátor
\begin{equation*}
    \Cop \coloneqq \sum_n \ket{n} \bra{n} \Parity,
\end{equation*}
ten komutuje s $\hat\G$ i $\Parity$ a tvoří základ skalárního součinu, pod kterým je $\ket{n}$ ortonormální systém:
\begin{equation*}
    (\psi, \phi)_\mathcal{CPT} \coloneqq \bra{\psi} \Parity \Cop \ket{\phi} .
\end{equation*}
Vypočtěte $\Cop$ a pro $a<K$ ověřte, že
\begin{enumerate}
    \item $\Cop^2 = \1$
    \item $\Cop \ket{\pm} = \pm \ket{\pm}$
    \item $\Cop \ket{+}\bra{+} \Parity + \Cop \ket{-}\bra{-} \Parity = \1$
    \item $\bra{m} \Parity \Cop \ket{n} = \delta_{mn}$
\end{enumerate}

\subsection{Řešení}
\begin{gather*}
    \Parity^2 = \smat[0.8]{1 \\& -1}^2 = \smat[0.8]{1\\&1}
    \\\\
    \Parity \; \hat\G \; \Parity =
    \smat[0.8]{1 \\& -1}
    \smat[0.8]{\ph{-}K & -\i a \\ -\i a & -K}
    \smat[0.8]{1 \\& -1}
    =
    \smat[0.8]{K & \ph{-} \i a \\ \i a & -K}
    = \hat \G^+
    \\\\
    0 = \left|\hat\G - \lambda\1\right| = \begin{vmatrix}
        K-\lambda & -\i a \\ -\i a & -K-\lambda
    \end{vmatrix}
    = -(K-\lambda)(K+\lambda) + a^2 = \lambda^2 - (K^2 - a^2)
    \\
    \lambda = \pm \sqrt{K^2 - a^2} \equiv \pm \G
    \\\\
    \ket{\pm} \propto \mat{
        \sqrt{ \left(\frac{K}{a}\right)^2 - 1 } \pm \frac{K}{a} \\
        \mp \i
    }
    \\\\
    C_\pm =
    \norm{ \smat[0.8]{
        \sqrt{ \left(\frac{K}{a}\right)^2 - 1 } \pm \frac{K}{a} \\
        \mp \i
    } }^2
    = \left| \sqrt{\left(\tfrac{K}{a}\right)^2 - 1} \pm \frac{K}{a} \right|^2 - 1
\end{gather*}
Normalizovat vektory $\ket\pm$ by nám nepřineslo žádnou výhodu, zapamatujeme si tedy pouze, jakým faktorem chceme případný výsledek dělit. Přistoupíme k výpočtu $\bra\pm \Parity \ket\pm$:
\begin{gather*}
    \bra\pm \Parity \ket\pm
    \propto \smat[0.8]{
        \operatorname{conj} \sqrt{ \left(\frac{K}{a}\right)^2 - 1 } \pm \frac{K}{a} &
        \;\mp \i
    }
    \smat[0.8]{1 \\& -1}
    \smat[0.8]{
        \sqrt{ \left(\frac{K}{a}\right)^2 - 1 } \pm \frac{K}{a} \\
        \mp \i
    }
    = \left| \sqrt{ \left(\tfrac{K}{a}\right)^2 - 1 } \pm \frac{K}{a} \right|^2 + 1
\end{gather*}
Pro $a>K$ si můžeme $a$ parametrizovat jako $a(t) = K \cosh t, \; t \in \R$. Potom nám vyjde:
\begin{gather*}
    \bra\pm \Parity \ket\pm
    \propto \left| \sqrt{\frac{1}{\cosh^2 t} - 1 } \pm \frac{1}{\cosh t} \right|^2 + 1
    = \left| \sqrt{ \frac{1 - \cosh^2 t}{\cosh^2 t} } \pm \frac{1}{\cosh t} \right|^2 + 1
\end{gather*}


\end{document}
